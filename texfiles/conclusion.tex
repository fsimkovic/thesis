\chapter{Conclusion \& Outlook}
\clearpage

\section{Conclusion}
The successful disentanglement of direct and indirect residue contacts in contact prediction revolutionised many aspects of Structural Bioinformatics research \cite{Simkovic2017-xs}. Successful applications of predicted contact information range from accurately defining domain boundaries \cite{Sadowski2013-zu} to identifying druggable protein-protein interfaces \cite{Bai2016-sw}. Although many such applications have been highlighted over the last few years \cite{Simkovic2017-xs}, few concerned the topic of \gls{mr} in X-ray crystallography. In \cref{chap:proof_of_principle,chap:rosetta_energy_functions,chap:alternate_abinitio_protocols,chap:ample_decoys,chap:ample_flib}, work was presented that made the some of the first attempts to apply predicted contact information to explore some of its applications in \gls{mr}.

The use of contact prediction in template-fee protein structure prediction allowed researchers to predict the structure of many previously unknown protein folds based on their sequence alone \cite[e.g.,][]{Marks2011-os,Michel2014-eg,Kosciolek2014-bt,Ovchinnikov2015-tn,Ovchinnikov2016-jj,Michel2017-xh,De_Oliveira2018-sg,Ovchinnikov2017-nd,Wang2017-rx}. The major benefit of adding such information was to reduce the conformational search space, which allowed more challenging folds to be sampled correctly. Work presented in \cref{chap:proof_of_principle,chap:rosetta_energy_functions,chap:alternate_abinitio_protocols} further confirmed such findings. More importantly, the presented results highlighted that the modelling algorithm ROSETTA is very sensitive to the way contact predictions are introduced into its folding protocol. Two important examples included the up-weighting of \textbeta-strand contacts and the choice of energy function used to ``reward'' satisfied contacts. Furthermore, work in \cref{chap:alternate_abinitio_protocols} highlighted that fragment-based structure prediction algorithms may no longer be essential for accurate structure prediction. CONFOLD2, a fragment-independent algorithm, predicts protein structures using secondary structure and contact information alone, which provided decoys of comparable accuracy to the state-of-the-art ROSETTA. However, it is worth reiterating that the sample sizes in \cref{chap:proof_of_principle,chap:rosetta_energy_functions,chap:alternate_abinitio_protocols} were small compared to similar but large-scale studies \cite[e.g.][]{Michel2017-xh,De_Oliveira2016-gj,Ovchinnikov2015-tn,Wang2017-rx}. Despite a broad agreement in results, findings presented in this work may be dataset-specific, and thus further experiments are crucial to independently validating the findings presented as part of this work.

Beyond the prediction of protein structures, a major focus of the presented research centred on the benefit of improved structure predictions in unconventional \gls{mr}. In line with prior expectations, better structure predictions yielded more \gls{mr} structure solutions. In particular, previous weaknesses of the AMPLE approach --- a target's chain length and fold --- were partially overcome with contact-guided structure predictions. Some examples for which structure solutions were obtained exceed 200 residues in chain length, whilst many others contain large portions of \textbeta-structure. Nevertheless, simply adding contact predictions to template-free protein structure prediction is not sufficient to solve all trialled targets. In part, this limitation resulted from a lack of precision of predicted contact information for some targets, since contact data depends significantly on the availability of divergent homologous sequences. Further research is also required to address new limitations in AMPLE resulting from suboptimal processing of much more native-like structure predictions. One approach, outlined in \cref{chap:ample_decoys}, explored the incorporation of contact information in the AMPLE processing pipeline to address the latter issue. Contact information was used to estimate the similarity of a predicted decoy to its native structure by means of scoring its long-range contact satisfaction \cite{Kosciolek2014-bt,De_Oliveira2016-gj,Adhikari2018-lj}. Exclusion of the worst decoys by this metric prior to clustering allowed more fine-grain sampling in AMPLE, which turned unsuccessful decoy sets into ones with which the native structure was solvable. However, key examples presented in \cref{chap:proof_of_principle,chap:rosetta_energy_functions,chap:alternate_abinitio_protocols} also highlighted the requirement for further developments in \gls{mr}-related software to enable the automatic detection and subsequent processing of AMPLE ensemble search models, which were correctly placed but are undetectable as structure solutions by current metrics.

A further topic of research concerned the use of supersecondary structure elements or subfolds as \gls{mr} search models. The default mode in AMPLE currently relies on computationally expensive template-free structure predictions. Since contact predictions reached sufficient quality for protein families with many known sequences, such information could be used to identify matching subfolds in other, unrelated protein structures. In \cref{chap:ample_flib}, a new hybrid approach demonstrated the successful implementation of such an idea. Although imperfect at this stage, several examples highlighted the successful identification of such subfolds and subsequently successful \gls{mr} structure solution. Tied to this idea may also be recent research that attempts to identify subfolds by means of matching a predicted contact map to those extracted from protein structures \cite{Buchan2017-ox,Ovchinnikov2017-nd}.

\section{Outlook}
In this thesis the first applications of predicted contact information in \gls{mr} were presented. Despite the already promising results, this area of research is still in its infancy and a great number of potentially promising routes remain unexplored \cite{Simkovic2017-xs}. 

Earlier studies by \textcite{Rigden2002-mf} and \textcite{Sadowski2013-zu} demonstrated the successful application of predicted residue contacts to identify domain boundaries. Although unexplored to-date, precise domain boundary predictions could be applied for better domain boundary definitions prior to template-free structure prediction to avoid sampling of terminal loops and linkers, and thus improve protein structure prediction quality. Furthermore, contact information was used to improve the AMPLE ensemble-generation pipeline with respect to identifying poorly predicted decoys. However, the AMPLE pipeline might additionally benefit from predicted contact information to drive the truncation procedure. For example, contact data could be used to rank individual residues by their contribution to a contact network, similar to \textcite{Parente2015-mv}, and truncation driven by the rank order or a hybrid score, which also includes the structural variance. Additionally, contact prediction might be used in the context of identifying alternative conformational states \cite{Hopf2012-zl,Jana2014-rw,Sfriso2016-ml,Morcos2013-ks,Sutto2015-ck}. AMPLE could exploit these to identify structurally conserved residues shared by both states, and thus truncate to this conserved core. Alternatively, AMPLE could attempt remodelling after successful disentanglement of state-dependent contact pairs and try both conformations separately as ensemble search models. \textcite{Simkovic2017-xs} outlined many further such applications of contact prediction in the field of Structural Biology. Ultimately, the precision of contact information improves daily with the increasing depth of sequence databases, thus enabling an ever-increasing number of applications with more precise outcomes. Furthermore, many more research groups start to identify the value in using predicted contact information in their own studies, and by means of pushing the boundaries new tools and applications are most likely going to emerge. 

Despite the vast space of unexplored applications, predicted residue contacts with perfect precision may never solve all current or future challenges in unconventional \gls{mr}. Despite the ability to limit the conformational space search in template-free protein structure prediction greatly, sampling of larger protein targets will always remain difficult unless energy functions and force fields become true representations of all properties found \textit{in vivo}. Furthermore, computational resources need to expand to allow many more sampling steps. Additionally, many protein targets exist in multiple conformations. Energy functions in fragment-based template-free protein structure prediction may always favour one such conformation over all others, which may make conventional or unconventional \gls{mr} very challenging. 

Beyond limitations in Bioinformatics software to facilitate the generation of search models for unconventional \gls{mr}, limits are also posed in the procedure of \gls{mr} itself. The most prominent limitation may be the resolution of the experimental data, and the proportion of the search model compared to the content of the crystallographic unit cell. SHELXE \cite{Thorn2013-le}, a popular and powerful algorithm to perform density modification and main-chain autotracing, is heavily limited by a lower resolution limit of 2.5\AA. Thus, \gls{mr} pipelines, such as AMPLE \cite{Bibby2012-lm} or ARCIMBOLDO \cite{Sammito2015-dv}, may not be able to automatically detect correctly placed search models due to the current dependence on associated software metrics. Furthermore, \gls{mr} is extremely challenging, if not impossible, when the scattering matter, i.e. a correctly placed search model, is particularly small in relation to the asymmetric unit content whilst the resolution of the experimental data is low. Heavily truncated AMPLE search models or other fragments may thus never suffice as \gls{mr} probes regardless of the associated similarity to the native structure.

Finally, AMPLE and similar unconventional \gls{mr} software pipelines try to enable \gls{mr} when one or more sufficiently similar structures are unavailable to derive the essential phase information. Despite the relative rarity of such a scenario \cite{Chandonia2017-vf}, it is essential to provide routes to structure solution when conventional approaches fail since those cases may often provide novel or unexpected findings. The current toolbox for unconventional \gls{mr} provides idealised fragments \cite{Sammito2015-dv,Thomas2015-wu,Jenkins2018-gf}, supersecondary structure motifs \cite{Sammito2013-ug}, and ensemble search models extracted from a diversity of different starting structures \cite{Rigden2018-zt,Bibby2012-lm,Bibby2013-cp}. The former two are usually target-independent, and thus limited by structural deviations between selected search probes and the target. In comparison, the latter depend much more on accurate and target-specific starting structures but provide a great alternative in lower resolution cases or scenarios whereby larger search models are required. Therefore, unconventional \gls{mr} requires a diversity of approaches to attempt structure solutions of the most challenging cases. AMPLE and its improvements through predicted residue contacts should therefore be considered an important tool in this set of approaches.
