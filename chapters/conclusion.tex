\section{Conclusion}
The successful disentanglement of direct and indirect residue contacts in contact prediction revolutionised many aspects of Structural Bioinformatics research \cite{Simkovic2017-xs}. Successful applications of contact information range from accurately defining domain boundaries \cite{Sadowski2013-zu} to identifying druggable protein-protein interfaces \cite{Bai2016-sw}. Although many such applications have been highlighted over the last few years \cite{Simkovic2017-xs}, few concerned the topic of \gls{mr} in X-ray crystallography. In this thesis, work was presented that made first attempts to apply contact information to explore some of its application spectrum in \gls{mr}.

The use of contact information in \textit{ab initio} protein structure prediction allowed researchers to predict the structure of many previously unknown protein folds based on their sequence alone \cite[e.g.,][]{Marks2011-os,Michel2014-eg,Kosciolek2014-bt,Ovchinnikov2015-tn,Ovchinnikov2016-jj,Michel2017-xh,De_Oliveira2017-sg,Ovchinnikov2017-nd,Wang2017-rx}. The major benefit of adding such information was to reduce the conformational search space, which allowed more challenging folds to be sampled correctly. Work presented in \cref{chap:proof_of_principle,chap:rosetta_energy_functions,chap:alternate_abinitio_protocols} further confirm such findings. More importantly, the presented results highlight that the modelling algorithm ROSETTA is very sensitive to the way contact information is introduced into the ROSETTA folding protocol. Two important examples include the up-weighting of \textbeta-strand contacts and the choice of energy function used to ``reward'' satisfied contacts. Furthermore, work in \cref{chap:alternate_abinitio_protocols} highlights that fragment-based structure prediction algorithms may no longer be essential for accurate structure prediction. CONFOLD2, a fragment-independent algorithm, predicts the protein structure using secondary structure and contact information alone, which provided models of comparable accuracy to the state-of-the-art ROSETTA.

Beyond the prediction of protein structures, a major focus of the presented research centred on the benefit of such improved structure predictions in unconventional \gls{mr}. In-line with prior expectations, better structure predictions yield more \gls{mr} structure solutions. In particular, previous weakpoints of the AMPLE routine --- the target chain length and fold --- can successfully be addressed with contact-guided structure predictions. Some examples for which structure solutions were obtained exceed 200 residues in chain length, whilst many others contain large proportions of \textbeta-structure. Nevertheless, simply adding contact information to \textit{ab initio} protein structure prediction is not sufficient to solve all trialled targets. Thus, further research, outlined in \cref{chap:ample_decoys}, explored one way of incorporating contact information in the AMPLE processing pipeline. Contact information was used to estimate the similarity of a predicted decoy to its native structure, by means of scoring its long-range contact satisfaction. Exclusion of the worst decoys by this metric prior to clustering allowed more fine-grain sampling in AMPLE, which turned unsuccessful decoy sets into ones from which the native structure can be elucidated.

A further topic of research concerned the use of supersecondary structure elements or subfolds as \gls{mr} search models. The default mode in AMPLE currently relies on computationally expensive \textit{ab initio} structure predictions. Since contact predictions have reached sufficient quality for protein families with many known sequences, such information could be used to identify matching subfolds in other, unrelated protein structures. In \cref{chap:ample_flib}, a new hybrid approach demonstrated the successful implementation of such an idea. Although imperfect at this stage, several examples highlighted the successful identification of such subfolds and subsequently successful \gls{mr} structure solution.

\section{Outlook}
In this thesis first applications of predicted contact information in \gls{mr} were presented. Despite the already promising results, this area of research is still in its infancy and a great number of potentially promising routes remain unexplored. Earlier studies by \textcite{Rigden2002-mf} and \textcite{Sadowski2013-zu} demonstrated the successful application of residue contacts to identify domain boundaries. Although unexplored to-date, precise domain boundary predictions could be applied for better domain boundary definitions in \textit{ab initio} structure prediction to avoid sampling of terminal loops and linkers, and thus improve protein structure prediction quality. Furthermore, contact information was used to improve the AMPLE pipeline with respect to excluding poorly predicted decoys. However, the AMPLE ensemble generation pipeline might additionally benefit from contact information to aid the driving of the truncation procedure. For example, contact data could be used to rank individual residues by their contribution to a contact network, similar to \cite{Parente2015-mv}, and truncation driven by the rank order. Additionally, contact prediction might be used in the context of identifying alternative conformational states \cite{Hopf2012-zl,Jana2014-rw,Sfriso2016-ml,Morcos2013-ks, Sutto2015-ck}, which AMPLE could exploit to identify conserved residues between both states and truncate to this conserved core, or attempt remodelling after successful disentanglement of state-dependent contact pairs and try both conformations separately as ensemble search models. Besides the application of contact data in protein structure prediction, other alternatives need to be considered too. Recently, first tools were developed to match predicted contact maps to ones extracted from protein structures \cite{Buchan2017-ox,Ovchinnikov2017-nd}. It might be of interest to investigate how search models, such as distant homologs, could be identified by sequence searches aided with contact map matching. 
