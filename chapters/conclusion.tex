\section{Conclusion}
The successful disentanglement of direct and indirect residue-residue contacts during the prediction process revolutionised many aspects of Structural Bioinformatics research \cite{Simkovic2017-xs}. Successful application of contact information ranges from the accurately defining domain boundaries \cite{Sadowski2013-zu} to identifying druggable protein-protein interfaces \cite{Bai2016-sw}. Although many such applications have been highlighted over the last few years \cite{Simkovic2017-xs}, few concerned the topic of \gls{mr} in X-ray crystallography. In this thesis work was presented that made first attempts to apply precise contact information to \gls{mr}.

The use of contact information in \textit{ab initio} protein structure prediction allowed researchers to predict the structure of many previously unknown protein folds based on their sequence alone. The major benefit of adding such information was to reduce the conformational search space, which allowed more challenging folds to be sampled correctly. Work presented in \cref{chap:proof_of_principle,chap:rosetta_energy_functions,chap:alternate_abinitio_protocols} further confirmed such findings. More importantly, the presented results highlighted that the modelling algorithm ROSETTA is very sensitive to the way contact information is introduced into the folding protocol. Two important examples include the up-weighting of \textbeta-strand contacts and the choice of energy function used to ``reward'' satisfied contacts. Furthermore, work in \cref{chap:alternate_abinitio_protocols} highlights that fragment-based structure prediction algorithms may no longer be essential for accurate structure prediction. CONFOLD2, a fragment-independent algorithm, predicts the protein structure using secondary structure and contact information alone, which provided models of comparable accuracy to the state-of-the-art ROSETTA.

Beyond the prediction of protein structures, a major focus of the presented research centred on the benefit of such improved structure predictions in unconventional \gls{mr}. In-line with prior expectations, better structure predictions yield many more \gls{mr} structure solutions. In particular, previous weakpoints of the AMPLE routine --- the target chain length and fold --- can be successfully addressed with contact-guided structure predictions. Some examples for which structure solutions were obtained exceed 200 residues in chain length, whilst many others contain large proportions of \textbeta-structure. Furthermore, work outlined in \cref{chap:ample_decoys} highlights a further application of contact information in the AMPLE processing pipeline. The use of contact information to evaluate the quality of predicted decoys proved to be a useful proxy to exclude poorly predicted decoys. Incorporation of this concept enabled AMPLE to generate ensemble search models to elucidate previously intractable structures.

One final topic of research concerned the use of supersecondary structure or subfolds as \gls{mr} search models. The default mode in AMPLE currently relies on coputationally expensive \textit{ab initio} structure predictions. Since contact predictions have reached sufficient quality for protein families with many known sequences, such information could be used to identify matching subfolds in other, unrelated protein structures. In \cref{chap:ample_flib}, a new hybrid approach demonstrated the successful implementation of such an idea. Although imperfect at this stage, several examples highlighted the successful identification of such subfolds and subsequently successful \gls{mr} structure solution.

% \section{Outlook}
% - subselection shows AMPLE not perfect
% - domain boundaries
% - model building (validation & help)
