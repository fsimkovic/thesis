\section{Introduction}

\subsection{General introduction}

\subsection{Macromolecular Crystallography}

\subsubsection{X-ray scattering}

X-rays are high energy photons part of the electromagnetic spectrum with a wavelength 0.1-10nm \cite{Rupp2010-nc}. X-rays can be described as packets of travelling electromagnetic waves, whose electric field vector interacts with the charged electrons of matter \cite{Rupp2010-nc}. Such interaction, typically termed scattering, results in the diffraction of the incoming wave, which X-ray crystallography relies on.

In its simplest form, scattering of X-ray radiation can be explained in the scenario of exposure to a single free electron. The resulting scattering can be classed as elastic (Thomson scattering) or inelastic (Crompton scattering) \cite{Rupp2010-nc}. The latter --- scattering that results in a loss of energy of the emitting photon due to energy transfer onto the electron --- does not contribute to discrete scattering, the type of scattering X-ray diffraction relies on. In comparison, Thomson scattering does not result in a loss of energy of the emitting photon. This has significant effects, the incoming pphoton emits with the same frequency causing the electron to oscillate identically further enhancing the signal. However, it is important to understand that the scattering function of an electron is non-isotropic since the scattered intensity emits strongest in forward and backward directions \cite{Rupp2010-nc}.

If we expand the example to include all electrons in an atom and expose the atom to X-ray radiation, our theory needs to be slightly expanded. Given thatone or more electrons in an atom are not free but orbit around the atom's nucleus in a stable and defined manner, the distribution of these electrons around the nucleus determines the scattering of the incoming X-ray photons. The distribution of scattered photon waves is thus an overall representation of the probability distributions of each electron in the atom and is refered to as electron density $\rho(\vec{r})$. In X-ray scattering, it suffices to approximate the shape of the electron density to a sphere. If we now consider the emitting wave $\vec{s_1}$ of an X-ray photon scattered by any position $\vec{r}$ in the electron density of an atom, then the phase difference $\Delta\varphi$ to the incoming wave $\vec{s_0}$ can be described by \cref{eq:phase_difference} \cite{Rupp2010-nc}. 

\begin{equation}
    \Delta\varphi=2\pi\left(\vec{s_1}-\vec{s_0}\right)\vec{r}=2\pi\vec{S}\vec{r}
    \label{eq:phase_difference}
\end{equation}

If more than one electron in an atom's electron density scatter the incoming X-ray wave, then the emitting partial waves can be described by the atomic scattering function $f_s$ (\cref{eq:atomic_scattering_factor}), which describes the interference of all scattered waves \cite{Rupp2010-nc}. The total scattering power of an atom is proportional to the number of electrons and element-specific with heavier atoms scattering more strongly. Given the approximation of a centrosymmetric electron density, the atomic scattering function is also symmetric. 

\begin{equation}
    f_s=\int\limits_{\vec{r}}^{V(atoms)}\rho\left(\vec{r}\right)e^{2\pi\\i\vec{S}\vec{r}}d\vec{r}
    \label{eq:atomic_scattering_factor}
\end{equation}

With an enhanced understanding of X-ray scattering of electrons in a single atom, it is important to consider X-ray scattering of adjacent atoms, such as it is typically found in molecules. If the electromagnetic wave of a X-ray photon excites all electrons of adjacent atoms, then the resulting partial waves result in constructive or destructive interference. Maximal interference can be obtained when all partial waves are in phase, and maximal destructive interference when out-of-phase. This leads to varying intensities of the emitting X-ray photon at different points in space. To obtain the overall scattering power $F_s$ of all contributing atoms, \cref{eq:atomic_scattering_factor} needs to be modified to include the sum over all atoms $j$ as described in \cref{eq:total_scattering_power}.

\begin{equation}
    F_s=\sum_{j=1}^{atoms}f_{s,j}^0e^{2\pi\\i\vec{S}\vec{r}_j}
    \label{eq:total_scattering_power}
\end{equation}

% adjacent atoms to single molecule
% single molecule to crystal


% \begin{equation}
%     \vec{S}\vec{a}=n_1; \vec{S}\vec{b}=n_2; \vec{S}\vec{c}=n_3
%     \label{eq:laue_equations}
% \end{equation}
%
% \begin{equation}
%     n\lambda=2d_{hkl}sin\theta
%     \label{eq:bragg_law}
% \end{equation}
%
% \begin{equation}
%     F_s=\sum_{j=1}^{atoms}f_{s,j}^0*e^{2\pi\\i\vec{S}\vec{r}_j}=\sum_{j=1}^{atoms}f_{s,j}^0*e^{2\pi\\i\vec{h}\vec{x}_j}
%     \label{eq:structure_factors}
% \end{equation}

