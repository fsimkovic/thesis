
\section{Dataset creation}

\subsection{FLUME dataset}
A test set if 21 globular protein targets was manually selected to include a range of chain lengths, fold architectures, X-ray diffraction data resolutions and \gls{msa} depths for contact prediction. The test set covered the three fold classes (\textalpha-helical, mixed \textalpha-\textbeta and \textbeta-sheet) and each target was grouped based on its secondary-structure content as defined by DSSP \cite{Kabsch1983-dy}. The chain length of the target sequences ranges from 62 to 221 residues and each crystal structure contained one molecule in the asymmetric unit. The resolutions of the crystal structures ranged from 1.0 to 2.3\AA. The FASTA sequences of each target, as provided by the RCSB \gls{pdb} (\url{www.rcsb.org}), were modelled, rather than the sequence that was visibly present in the crystallographic model. 

\subsection{KEENO dataset} \label{sec:methods_keeno_dataset}
An unbiased selection of 27 non-redundant protein targets was selected using the following protocol.

The Pfam v29.0 \cite{Finn2016-tz} database was filtered for all protein families with at least one representative structure in the RCSB \gls{pdb} \cite{Berman2000-qj} database. Each representative had to have monomeric protein stoichiometry and its fold classified in the SCOPe v2.05 database \cite{Chandonia2017-bd}. Targets with fold assignments other than "a" (all-\textalpha), "b" (all-\textbeta), "c" (mixed \textalpha+\textbeta) or "d" (mixed \textalpha/\textbeta) were excluded to exclusively focus on regular globular protein folds. Each resulting protein target was screened against the RESTful API of the RCSB \gls{pdb} (\url{www.rcsb.org}) webserver to identify targets meeting the following criteria: experimental technique is X-ray crystallography; chain length is $\geq100$ residues and $\leq250$ residues; resolution is between 1.3 and 2.3\AA; structure factor amplitudes are deposited in the Protein Data Bank \cite{Berman2000-qj} database; and there is only a single molecule in the asymmetric unit. The resulting protein structures were cross-validated against the \acrlong{pdb} of Transmembrane Proteins \cite{Tusnady2005-lp} webserver to exclude any possible matches. Subsequently, one representative entry was randomly selected for each Pfam family.

The final set of 27 non-redundant targets was determined using further target characterisation and grouping of Pfam families. All targets were grouped using three criteria: domain fold, target chain length and alignment depth. The former consisted of the three fold classes all-\textalpha, all-\textbeta, and mixed \textalpha-\textbeta\ (\textalpha+\textbeta\ and \textalpha/\textbeta) and targets were group using the SCOPe assignment. The target chain lengths were obtained from the deposited information via the RESTful API of the RCSB \gls{pdb} web server and split into three bins, using 150 and 200 residues as bin edges. Furthermore, the alignment depth was calculated for the sequence alignment of each Pfam family and three bins established with bin edges of 100 and 200 sequences. Thus, all targets were classed in three bins for each of the three features. 

The final selection of the 27 targets was performed by randomly selecting one target for each feature combination. To ensure even sampling across the three different fold categories, a target function was employed to identify roughly even target characteristics in each group. The alignment depth and chain length were used as metrics, and had to be within $\pm15$ units to the values of the other fold classes. This created two conditions that had to be met for a randomly chosen sample to be accepted.