\section{Introduction}
Work presented in previous chapters highlighted the much improved \textit{ab initio} decoy quality achievable by restraining the conformational search space with residue-residue contact information. Furthermore, the data also highlighted that this improvement extends AMPLE's tractability of achieving structure solution for more challenging targets. However, the data also indicated that AMPLE's current protocol is not tailored towards decoy sets with overall much higher accuracy. Decoy sets with correctly predicted folds (average \gls{tmscore} $>0.5$ per 1,000 decoys) did not generate any or many ensemble search models leading to \gls{mr} structure solution. It also became apparent that certain decoy sets contained few high-quality decoys that were lost in the process of clustering, since non of the top-10 SPICKER clusters contained that fold.

Beyond the limitations observed in AMPLE, \textit{ab initio} decoy similarity in exceptional cases approaches a near-identical fold (RMSD $<1.5$\AA) to the crystallised one. Although challenging by current means to identify these decoys, it is of great interest since these decoys might be sufficient by themselves as \gls{mr} search models. Contact information, which was used to restrain the folding protocol, might provide enough information to drive such filtering. Indeed, \textcite{De_Oliveira2017-gj} found that long-range residue-residue contact pair satisfaction correlates well with decoy quality. Additionally, \textcite{Adhikari2018-lj} use long-range contact satisfaction routinely in CONFOLD2 to exclude the worst decoys amongst the set predicted ones.

Thus, this chapter focuses on exploring alternative strategies of decoy selection in AMPLE, and if contact information can be used beyond the distance-restraint application in \textit{ab initio} protein structure prediction.

\section{Materials \& Methods}
\subsection{Target selection}
The dataset for this study consisted of 113 ROSETTA decoy sets generated throughout the works outlined in previous chapters. The 113 decoy sets covered all targets in the ORIGINAL (\cref{table:appendix_dataset_original}), PREDICTORS (\cref{table:appendix_dataset_predictors}) and TRANSMEMBRANE (\cref{table:appendix_dataset_transmembrane}) datasets. Top-$L$ ($>5$ residues sequence separation) CCMPRED \cite{Seemayer2014-zp}, PCONSC2 \cite{Skwark2014-qp}, METAPSICOV STAGE 1 \cite{Jones2015-vq} and MEMBRAIN \cite{Yang2013-bf} contact pairs were used in combination with the \textit{FADE} energy function to restrain the \textit{ab initio} structure prediction process.

\subsection{Computation of range-specific satisfaction scores}
The satisfaction of short- ($>6$ residues sequence separation), medium- ($>12$ residues sequence separation) and long-range contact pairs ($>23$ residues sequence separation; see \cref{sec:methods_longrange_satisfaction}) were computed for each decoy in each set. Hereby, the contact pairs of the original set of contact pairs used to restrain the \textit{ab initio} structure prediction protocol were extracted, matched against the contact pairs extracted from individual decoys and the contact pair range-specific satisfaction score evaluated. 

\subsection{Decoy subselection}
Each set of decoys was then ranked in descending order by their long-range contact pair satisfaction scores and the $n$ decoys with the lowest scores removed from each set. The number of decoys to remove $n$ were selected using a number of different strategies:

\begin{itemize}
    \item \textit{NONE}: leave the original set unchanged
    \item \textit{LINEAR}: remove the worst 500 decoys
    \item \textit{CUTOFF}: remove all decoys with a score of $<0.287$ 
    \item \textit{SCALED}: remove all decoys with a scaled score of $<0.5$, where the scaled score is score divided by set average
\end{itemize}

The fixed definition in the \textit{CUTOFF} strategy was determined by \textcite{De_Oliveira2017-gj}. The scaled score used by the \textit{SCALED} strategy was computed by dividing each decoy's long-range contact pair satisfaction by the set's average.

\subsection{Molecular Replacement}
To evaluate the benefits of such subselection to \gls{mr} in AMPLE, a subset of 35 decoy sets (spanning 35 unique targets) were processed as described above and subjected to AMPLE v1.2.0 and CCP4 v7.0.28. Default options were chosen with few exceptions: decoys in all 10 clusters were used, subcluster radii thresholds were set to 1 and 3\AA, and side-chain treatments were set to \texttt{polyala} only. This change in protocol was shown to be advantageous in most cases by Jens Thomas (PhD Thesis), and thus trialled in this context. 

To allow comparability of these results to previous AMPLE runs, an additional condition was added, namely \textit{NONE\_classic}. The decoy set from the \textit{NONE} strategy was hereby subjected to the AMPLE protocol with default settings.

Each \gls{mr} run was assessed using the criteria defined in \cref{sec:methods_mr_success}.

\section{Results}


% - Is there a correlation between contact satisfaction and model quality?
% --- strong positive correlation demonstrated by \textcite{Kosciolek2014-bt} although performed on PSICOV dataset (150 small globular targets)
% - Does this correlation hold for all targets or a particular fold only?
% - Are there situations where the correlation is non-existant or negative? 

% - How many decoys do we exclude for the different selection criteria?
% - How does the correlation affect the selection of decoys?
% - How do the exclusion conditions affect the quality measures of our decoy sets?
% - How does the selection impact AMPLE ensemble search model generation?
% - How does decoy exclusion impact MR?
