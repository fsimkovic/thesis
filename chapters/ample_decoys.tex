\section{Introduction}
% - something about high decoy accuracy but little success
% - identification of best decoys challenging and long-range contact pairs useful
% - attempt to remove worst decoys to improve overall set accuracy

\section{Materials \& Methods}
\subsection{Target selection}
The dataset for this study consisted of 35 ROSETTA decoy sets generated in previous chapters. The 35 decoy sets covered all targets in the ORIGINAL (\cref{table:appendix_dataset_original}) and TRANSMEMBRANE (\cref{table:appendix_dataset_transmembrane}) datasets. Top-$L$ (min. sequence separation of 5 residues) CCMPRED \cite{Seemayer2014-zp} and PCONSC2 \cite{Skwark2014-qp} contact pairs were used in combination with the \textit{FADE} energy function to restrain the \textit{ab initio} structure prediction process (see \textcolor{red}{Chapter with PCONSFOLD}).

\subsection{Decoy subselection}
The precision of long-range contact pairs ($>23$ residues sequence separation; see \cref{sec:methods_longrange_precision}) was computed for each decoy in each set. Hereby, the long-range contact pairs of the original set of contact pairs used to restrain the \textit{ab initio} structure prediction protocol were extracted, and their precision evaluated with reference to the predicted decoy. This long-range contact pair precision thus quantifies the satisfaction of long-range contact pairs.

Each set of decoys was then ranked in descending order by their long-range contact pairs precision and the $n$ decoys with the lowest scores removed from the set. The number of decoys to remove $n$ were selected using a number of different strategies:

\begin{itemize}
    \item \textit{NONE}: leave the original set unchanged
    \item \textit{LINEAR}: remove the worst 500 decoys
    \item \textit{CUTOFF}: remove all decoys with a score of $<0.287$ 
    \item \textit{SCALED}: remove all decoys with a scaled score of $<0.5$, where the scaled score is score divided by set average
\end{itemize}

The fixed definition in the \textit{CUTOFF} strategy was determined by \textcite{De_Oliveira2017-gj}. The scaled score used by the \textit{SCALED} strategy was computed by dividing each decoy's long-range contact pair precision by the set's average.

\subsection{Molecular Replacement}
The decoy sets generated in the previous step were subjected to AMPLE v1.2.0 and CCP4 v7.0.28. Default options were chosen with few exceptions: decoys in all 10 clusters were used, subcluster radii thresholds were set to 1 and 3\AA, and side-chain treatments were set to poly-Ala only. This change in protocol was shown to be advantageous in most cases by Jens Thomas (PhD Thesis), and thus trialled in this context. 

To allow comparability of these results to previous AMPLE runs, an additional condition was added, namely \textit{NONE\_classic}. The decoy set from the \textit{NONE} strategy was hereby subjected to the AMPLE protocol with default settings.

Each \gls{mr} run was assessed using the criteria defined in \cref{sec:methods_mr_success}.

