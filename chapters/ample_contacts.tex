\textbf{Note: }\textit{The majority of the work presented in this chapter was published in two independent pieces of work. All work relating to the globular targets was published by \textcite{Simkovic2016-wk}, and a great majority of work relating to the transmembrane targets by \textcite{Thomas2017-sh}. As such, this chapter consists of extracts from both publications with additional information where appropriate.}

\section{Materials \& Methods}
\subsection{Target selection}
\subsubsection{ORIGINAL dataset} \label{sec:methods_dataset_original}
A test set of 21 globular protein targets was manually selected to include a range of chain lengths, fold architectures, X-ray diffraction data resolutions and \gls{msa} depths for contact prediction  (\cref{table:appendix_dataset_original}). The test set covers the three fold classes (\textalpha-helical, mixed \textalpha-\textbeta\ and \textbeta-sheet) and targets were grouped using their DSSP \cite{Kabsch1983-rr} secondary-structure assignment. Target chain lengths fall in the range of [62, 221] residues. Each crystal structure contains one molecule per asymmetric unit and the resolutions of the experimental data is in range from 1.0 to 2.3\AA.

\subsubsection{TRANSMEMBRANE dataset} \label{sec:methods_dataset_transmembrane}
The selection of this dataset was done by \cite{Thomas2017-sh}. In summary, 14 non-redundant transmembrane protein targets were selected from the \gls{pdbtm} \cite{Tusnady2005-ns}, with a chain length of $<250$ residues and resolution of $<2.5$\AA. The final selection is summarised in \cref{table:appendix_dataset_transmembrane}.

\subsection{Contact prediction}
\subsubsection{Globular targets}
One contact map was predicted for each of the 21 targets with the fully automated metapredictor PCONSC2 \cite{Skwark2014-qp}. In summary, each \gls{msa} was generated with each of JACKHMMER  \cite{Johnson2010-uz} against the \texttt{uniref100} database and HHBLITS v2.0.15 \cite{Remmert2011-kt} against the \texttt{uniprot20} database v2013.03 \cite{Bateman2017-pb} at E-value cutoffs of 10\textsuperscript{-40}, 10\textsuperscript{-10}, 10\textsuperscript{-4} and 1. Each \gls{msa} was analysed with PSICOV \cite{Jones2012-ks} and PLMDCA \cite{Ekeberg2014-kf} to produce 16 individual sets of contact predictions. All 16 predictions, combined with a secondary structure prediction, solvent accessibility information and sequence profile were then provided to a deep learning algorithm \cite{Skwark2014-qp} to identify protein-like contact patterns. The latter produced a final contact map for each target sequence. 

An additional contact map for \textbeta-structure containing targets was predicted using CCMPRED \cite{Seemayer2014-zp} and reduced to \textbeta-sheet contact pairs using the CCMPRED-specific filtering protocol BBCONTACTS \cite{Andreani2015-qn}. Each \gls{msa} for CCMPRED contact prediction was obtained using HHBLITS v2.0.15 \cite{Remmert2011-kt}. This entailed two sequence search iterations with an E-value cutoff of 10\textsuperscript{-3} against the \texttt{uniprot20} database v2013.03 \cite{Bateman2017-pb} and filtering to 90\% sequence identity using HHFILTER v2.0.15 \cite{Remmert2011-kt} to reduce sequence redundancy in the \gls{msa}. Besides the contact matrix as input, BBCONTACTS requires a secondary structure prediction and a factor describing the range of predicted contacts in the \gls{msa}. The latter was shown to depend on the sequence count in the \gls{msa} ($N$) and the target chain length ($L$). Thus, the factor describing this \gls{msa}-specific diversity was calculated using \cref{eq:bbcontacts_diversity} \cite{Andreani2015-qn}. 

\begin{equation}
    \eta=N/L 
    \label{eq:bbcontacts_diversity}
\end{equation}
\equations{BBCONTACTS sequence alignment diversity factor}

The secondary structure for each sequence was predicted using the \texttt{addss.pl} \cite{Remmert2011-kt} script distributed with HHSUITE v2.0.16 \cite{Soding2005-hw}. Hereafter the term BBCONTACTS will be used to describe the full process from the target sequence to the filtered \textbeta-strand contact map. At no point do contact prediction algorithms use structural information from structurally characterised proteins.

\subsubsection{Transmembrane targets}
For each target, a \gls{msa} was generated using the HHBLITS v2.0.16 \cite{Remmert2011-kt} against \texttt{uniprot20} database v2016\_02 \cite{Bateman2017-pb}. A contact meta-prediction using METAPSICOV v1.04 \cite{Jones2015-vq} was generated, which in turn used the contact prediction algorithms CCMPRED v0.3.2 \cite{Seemayer2014-zp}, FREECONTACT v1.0.21 \cite{Kajan2014-bx} and PSICOV v2.1b3 \cite{Jones2012-ks}. The predictions from METAPSICOV STAGE 1 were used and STAGE 2 discarded since the former yields more accurate structure predictions \cite{Jones2015-vq}. The CCMPRED predictions generated by METAPSICOV were used as the CCMPRED predictions. A set of contacts was also generated using the MEMBRAIN server (\href{http://www.csbio.sjtu.edu.cn/bioinf/MemBrain}{http://www.csbio.sjtu.edu.cn/bioinf/MemBrain}) v2015-03-15 \cite{Yang2013-bf}.

\subsection{Contact-to-restraint conversion}
\subsection{\textit{Ab initio} structure prediction}
\subsection{Molecular Replacement in AMPLE}
