% \section{Introduction}
%
% - \texit{Ab initio} structure prediction is a balancing act between time spent to generate decoys and the quality of the final decoy set
% - \texit{Ab initio} structure prediction can only predict accurate structures with a subselection of fragments collective capturing the overall target fold
%
% - AMPLE requires decoys with the overall accurate fold yet enough diversity in the set to truncate to the conserved core
% - AMPLE typically succeeds with small to medium size fragments relative to the target sequence (20-50\%)
%
% This bares the question whether fragments extracted for \texit{ab initio} structure prediction are sufficient as MR
% search models. This stems primarily on the assumption that fragment picking algorithms can identify fragments based on
% sequence features that are structurally similar to a sequence position in our target.
%
% - Something about the addition of contacts as unique feature for further identifying correct fragments
% - Something about differences to other fragment-MR approaches

\section{Methods}

\subsection{Target selection}
Targets were manually chosen using favourable cases for this proof-of-principle study, i.e. resolution was chosen to
be ~1.5\AA, target chain lengths were $<150$ residues, and only a single molecule is present in the asymmetric unit
(Table \ref{table:ample_flib_target_properties}).

\begin{table}[H]
  \centering
  \begin{tabularx}{\textwidth}{|X|X|X|X|X|X|}
      \hline
      \textbf{Target} & \textbf{Fold} & \textbf{Chain Length} & \textbf{Resolution (\AA)} & \textbf{Nmol/ASU} &
\textbf{Space Group} \\ 
      \hline
      1aba & mixed \textalpha-\textbeta & 87    & 1.45 & 1      & P $2_1$ $2_1$ $2_1$   \\ \hline
      1lo7 & mixed \textalpha-\textbeta & 141   & 1.50 & 1      & I $2$ $2$ $2$         \\ \hline
      1u06 & all-\textbeta              & 62    & 1.49 & 1      & P $2_1$ $2_1$ $2_1$   \\ \hline
      5nfc & all-\textbeta              & 147   & 1.59 & 1      & P $2_1$ $2_1$ $2_1$   \\ \hline
  \end{tabularx}
  \caption{Target properties.}
  \label{table:ample_flib_target_properties}
\end{table}

% \subsection{Preparation for Flib fragment picking}
% - Secondary structure prediction with PSICOV
% - Contact prediction with METAPSICOV
% - HHpred with custom script
% - BLASTp homolog identification

\subsection{Fragment picking using Flib}
Fragments for this study were picked using the Flib algorithm \cite{De_Oliveira2015-ba}. 

Two modifications were made to the default Flib v1.01 (\url{https://github.com/sauloho/Flib-Coevo}) protocol. The
first focuses on exclusion of fragments with $>90$\% helical content (assigned by DSSP \cite{Frishman1995-ns}). The second modification
was to allow fragments with R.M.S.D. $>10.0$\AA\ to be considered.

Two-hundred fragments were picked per target sequence position. Top-$L$ or $L/2$ contact pairs were considered from both
METAPSICOV STAGE 1 and STAGE 2 predictions with a minimum sequence separation of either 6 or 12 residues. Helical
fragments were either in- or excluded and only fragments with length between 6 or 12 residues up-to 63 residues
considered. Overall, this generated 16 fragment libraries per target. 

Each fragment library was then filtered to remove homologs. Hereby, BLASTp \cite{Altschul1990-nc} and HHpred
\cite{Soding2005-sx} searches were conducted to identify homologous PDB entries. The BLASTp search was performed identically to \cite{De_Oliveira2015-ba}. The HHpred search parameters were identical to the MPI-Toolkit \cite{Biegert2006-ny} webserver version (\url{https://toolkit.tuebingen.mpg.de/}). Fragments derived from PDB entries identified by BLASTp and HHpred (probability score of $\geq20.0$) were excluded from the fragment libraries.

All fragments per target were then grouped by their length. Subsequently, they were ranked twice by Flib scores and
R.M.S.D. values, and the best fragment selected. The coordinates for the selected fragments were then extracted and two
versions for each file created, once containing backbone atoms only and once containing all atoms.

\subsection{Molecular Replacement in MrBUMP}
The previously extracted fragments were subjected to the MR pipeline MrBUMP \cite{Keegan2008-hk}. The latter uses PHASER
\cite{McCoy2007-bf} for MR, REFMAC5 \cite{Murshudov2011-we} for refinement and SHELXE \cite{Thorn2013-ir} for density modification and main-chain tracing. MrBUMP default parameters were used with exception of the PHASER RMS estimate. MR was attempted for each coordinate file with a PHASER RMS value of 0.1, 0.6, and 1.0.

MR success was assessed identically to Chapter XYZ.
